\documentclass[12pt, a4paper]{article}
\usepackage[utf8]{inputenc}
\usepackage{lipsum}
\usepackage{float}
\usepackage{tikz}
\usepackage{framed}
\usepackage{verbatim} %multiline comment
\usepackage{fullpage}
\usepackage[super]{nth}
\usepackage{amsmath}
\usepackage{appendix}

% /today-setup
\usepackage[UKenglish]{babel}
\usepackage[useregional]{datetime2}
\DTMlangsetup[en-GB]{ord=raise,monthyearsep={,\space}}
\DTMsavedate{sample}{2017-01-31}

% zero-indexed sections
\setcounter{section}{-1}

\usepackage{hyperref}
\hypersetup{
	%pdfpagelabels=true,% is supposedly already set
	plainpages=false,%
	pdfauthor={Author(s)},%
	pdftitle={Title},%
	pdfsubject={Subject},%
	bookmarksnumbered=true,%
	colorlinks=true,%
	citecolor=black,%
	filecolor=black,%
	linkcolor=black,% you should probably change this to black before printing
	urlcolor=black,%
	pdfstartview=FitH%
}

\usepackage[newfloat]{minted}
\usepackage{caption}
\usemintedstyle{tango}
%minted-specific setup
%default style
\setminted[]{
    breaklines,
    frame=lines,
    framesep=2mm,
    baselinestretch=1.2,
    fontsize=\footnotesize,
    linenos
}
\newenvironment{code}{\captionsetup{type=listing}}{}
\SetupFloatingEnvironment{listing}{name=Code block}

\usepackage{geometry}
 \geometry{
 a4paper,
 top=20mm,
 bottom=20mm,
 }


\begin{document}

\section{The Masking Tape Game}
This game was developed during a slow Thursday in October, 2019, by a Computer Science group whom shall remain nameless. 

This game and all documentation is released under \href{https://www.gnu.org/licenses/fdl.tex}{GNU Free Documentation License (\url{https://www.gnu.org/licenses/fdl.tex})}\footnote{Contributions are welcome at \url{https://github.com/cogitantium/masking-tape-game}}.

\section{Prerequisites}
\begin{itemize}
    \item 1 roll of masking tape
    \begin{itemize}
        \item May be partially used to introduce variance
        \item Brand \textit{tesa\textsuperscript{\textregistered}} works well
    \end{itemize}
    \item 1 marker which works on masking tape\footnote{Ball-point pens do not work well}
\end{itemize}

\section{Rules}
\begin{enumerate}
    \item The turn-order must be decided by the algorithm described in section \ref{turn_alg}\footnote{If you can run the algorithm in your head, then that's fine}
    \item The roller must obey the following conditions:
    \begin{enumerate}
        \item Any roll must be thrown from behind the back line
        \item Any roll must begin rolling within two meters of the back line\footnote{This is to avoid yeeting the roll - instead of rolling the roll}
        \item Underhand throws are the only kind of throws allowed
        \item A piece of tape with the name of the participant is placed on the final position
        \begin{enumerate}
            \item The final position is not necessarily the furthest position. If the roll bounces back from an obstacle then that resulting position is the final position.
        \end{enumerate}
    \end{enumerate}
    \item Participants waiting on their turn must not jeer or otherwise impact the performance of the roller
    \item If a participant blocks the rolling roll, rolled by the roller, their mark is reset 
    \item The winner of the game is the participant who has their mark the furthest from the back line
    \item The back line is marked by red masking tape near the door-frame of room 1.221 in Novi 9, SLV 312
    \item Variations on these rules that introduce so-called \textit{drinking-rules} must be submitted by PR before being considered valid\footnote{These rules need not be approved nor merged with \texttt{master} before being valid}
\end{enumerate}

\pagebreak
\appendix

\section{Turn-algorithm}\label{turn_alg}
The turn-order is based on the lexicographic order of the reversed SHA256-hash of each participant's name.
\begin{code}
\begin{minted}{python}
(lambda hashlib, sys: [(i == 0) and bool(print(f" #:         NAME | HASH\n {''.join(['=' for i in range(25)])}")) or bool(print(x)) for i, x in enumerate([f"{i:>2}: {x[0]:>12} | {x[1][:7]}" for i, x in enumerate(sorted([(name, bool(sha.update(name.encode("UTF-8"))) or str(sha.hexdigest())[::-1]) for name, sha in [(name, hashlib.new("sha256")) for name in sys.argv[1:]]], key=lambda x: x[1]))])])(__import__("hashlib"), __import__("sys"))
\end{minted}
\end{code}

Naturally, this snippet is self-documenting. On the following input\footnote{Note that names of participants may not contain spaces. Instead, use an agreed-upon delimiter.} \mint{bash}|python gen.py foo bar foo_bar john hüttel rené_hüttel|

The resulting table of turn-order is generated:

\begin{code}
\begin{minted}{text}
 #:         NAME | HASH
 =========================
 0:       hüttel | 3e18a42
 1:          bar | 9bf8fbf
 2:         john | a706cec
 3:      foo_bar | ab517ce
 4:  rené_hüttel | ca7f318
 5:          foo | ea7e662
\end{minted}
\end{code}

To no surprise, Hüttel is the participant to start. 


\end{document}
